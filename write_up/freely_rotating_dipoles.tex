%\documentclass[12pt]{amsart}
\documentclass[prb,preprint]{revtex4-1} 
%Manuscripts that demonstrate new relations between apparently unrelated areas of physics are appropriate.
\usepackage{geometry} % see geometry.pdf on how to lay out the page. There's lots.
\usepackage{mathtools}
\geometry{a4paper} % or letter or a5paper or ... etc
% \geometry{landscape} % rotated page geometry

% See the ``Article customise'' template for come common customisations

%\date{} % delete this line to display the current date

%%% BEGIN DOCUMENT
\begin{document}

\title{Dynamics of Two Freely Rotating Dipoles}
\author{Peter Haugen}

\maketitle
%\tableofcontents

\section{Abstract}
	The equations of motion for two spherical dipoles moving freely in a plane are arrived at. Special consideration is given to the contact state. Their equilibria determined, small amplitude motion examined and large amplitude motion investigated to reveal exclusively quasi-periodic motion is possible.
	
\section{Introduction}
A system comprised of two unrestricted spherical magnetic dipoles presents an interesting and fundamental set of phenomenon as two magnetic dipoles is amongst the simplest non-trivial macroscopic magnetic systems. Understanding their dynamics could prove useful in contexts such as the large dipole field proposed to provide a solar wind umbrella for mars or small magnetic toothless-gears.

\section{Describing the System}
Intrinsic properties of the two dipoles are: their respective radii, which we shall label as $a_1$ and $a_2$, their masses, $m_1$ and $m_2$, and magnitudes of their dipole moments, $\mu_1$ and $\mu_2$.
\subsection{Coordinates}
\subsubsection{Cartesian}

Each of the dipoles has a location constrained to the x-y plane $\vec r_1$ and $\vec r_1$ along with an orientation for the dipole $\phi_1$ and $\phi_2$ which we will also constrain to the x-y plane, measured from the x-axis.
\subsubsection{Center Of Mass}
We can define a set of composite coordinates and quantities more appropriate for analyzing two-body system using the aforementioned independent coordinates a la Taylor's mechanics. 

The total and reduced mass

\begin{equation}
m_t = m_1+m_2
\qquad
m_r = \frac{m_1m_2}{m_1+m_2}
\end{equation}

The center of mass 
\begin{equation}
\vec R = \frac{m_1 \vec r_1+m_2 \vec r_2}{m_t}
\end{equation}

And the displacement of dipole 2 from dipole 1
\begin{equation}
\vec r 
=  \vec r_2-\vec r_1 
= r (\cos(\theta) \hat x+\sin(\theta) \hat y)
\end{equation}


\subsection{Hamiltonian}
To take advantage of the numerous analytic techniques that can be applied to a system's Hamiltonian, it must first be calculated. Which in turn requires that we find it's kinetic and potential energy in terms of it's coordinates.
\subsubsection{Energies}
The kinetic energy, T, in the center of mass coordinates can be found to be
%citing taylor mechanics for the reduced kinetic
\begin{equation}
T = \frac{1}{2}(
	m_t \dot R^2
	+m_r \dot r^2
	+m_r r^2 \dot \theta^2
	+ I_1 \dot \phi_1^2
	+ I_2 \dot \phi_2^2
)
\end{equation}
The potential energy for dipole 1 is merely it's dot product with the local field, which in this case is caused entirely by dipole 2.
\begin{equation}
U_1 = -\vec \mu_1 \cdot \vec B_2
\end{equation}

With the magnetic field for a dipole at some point of interest $\vec r_i$ being given by

\begin{equation}
\vec B_2 = 
\frac{\mu_0}{4\pi}(
	3\frac{\vec \mu_2 \cdot (\vec r_i - \vec r_2)}{|\vec r_i - \vec r_2|^5}(
	\vec r_i - \vec r_2
	)
	-\frac{\vec \mu_2}{|\vec r_i - \vec r_2|^3}
)
\end{equation}

Using our definition for $\vec r$ we can simplify the potential down to

\begin{equation}
U_1 = 
\frac{\mu_0}{4\pi}
\frac{1}{r^3}(
	\vec \mu_1 \cdot \vec \mu_2
	-3(
		\vec \mu_2 \cdot \hat r
		)(
		\vec \mu_1 \cdot \hat r		
		)
)
\end{equation}

The $\vec \mu_1 \cdot \vec \mu_2$ term accounts for the obvious relation the dipole orientation of the two dipoles has with $U$, and the $\vec \mu_i \cdot \hat r$ term accounts for the less obvious relationship that their geometric orientation with each other has on $U$ .
This is obviously symmetric under substitution of 1 for 2 and vice versa so we have a total interaction potential of

\begin{equation}
U =
U_1+U_2
= 
2U_1
\end{equation}

Which when expanding the dot products out in terms of all three respective angles leaves us with

\begin{equation}
  \begin{multlined}
\hat \mu_i =  \cos(\phi_i)\hat x + \sin(\phi_i)\hat y,
\hat r =  \cos(\theta)\hat x + \sin(\theta)\hat y
\\
U(\phi_1, \phi_2, \theta, r) =
-\frac{\mu_0}{4\pi}
\frac{\mu_1 \mu_2}{2}
\frac{1}{r^3}(
	\cos(\phi_1-\phi_2)
	+3\cos(\phi_1+\phi_2 -2\theta)
)
  \end{multlined}
\end{equation}

Having an expression for kinetic and potential energy, the Lagrangian is as follows.

\begin{equation}
  \begin{multlined}
    L=T-U=
    \frac{1}{2}(
        m_t \dot R^2
        +m_r \dot r^2
        +m_r r^2 \dot \theta^2
        + I_1 \dot \phi_1^2
        + I_2 \dot \phi_2^2
    )
    \\
    +
    \frac{\mu_0}{4\pi}
    \frac{\mu_1 \mu_2}{2}
    \frac{1}{r^3}(
        \cos(\phi_1-\phi_2)
        +3\cos(\phi_1+\phi_2 -2\theta)
    )
  \end{multlined}
\end{equation}

\subsubsection{Momenta}

The first observation we make is that the Lagrangian is independent of the center of mass position making the corresponding momenta a constant of motion and we will proceed assuming we are in the inertial frame where it is 0 and R is 0. Determining the other momenta we use the relationship $\partial_{\dot q_i} L =p_i$ to arrive at

\begin{subequations}
    \begin{equation}
        \partial_{\dot \phi_1} L = I_1\dot \phi_1 = p_{\phi_1}
    \end{equation}
    \begin{equation}
        \partial_{\dot \phi_2} L =I_2\dot \phi_2 = p_{\phi_2}
    \end{equation}
    \begin{equation}
        \partial_{\dot \theta} L =m_r r^2 \dot \theta = p_{\theta}
    \end{equation}
    \begin{equation}
        \partial_{\dot r} L = m_r \dot r = p_r
    \end{equation}
\end{subequations}

Allowing us to find the Hamiltonian with the expression of 

\begin{equation}
H=\Sigma_i p_i \dot q_i - L
=
\frac{1}{2}(
	\frac{p_r^2}{m_r}
	+\frac{p_\theta^2}{m_r r^2}
	+\frac{p_{\phi_1}^2}{I_1}
	+\frac{p_{\phi_2}^2}{I_2}
)+U(\phi_1,\phi_2,\theta, r)
\end{equation}
\subsection{Dimensionless coordinates}
To get at the essentials of the system we will examine it in a natural set of dimensions. First we will characterize the second dipole in terms of the first such that 
$m_2=\alpha m_1$,   
$a_2=\beta a_1$,
$\mu_2=\gamma \mu_1$. This lets us rewrite the energies as
%I = 2/5 mr^2
\begin{subequations}
    \begin{equation}
        T=\frac{1}{2}\left (
	\frac{(1+\alpha)m_1}{\alpha m_1^2} p_r^2
	+\frac{(1+\alpha)m_1}{\alpha m_1^2} \frac{p_\theta^2}{r^2}
	+p_{\phi_1}^2 \frac{5}{2m_1a_1^2}
	+p_{\phi_2}^2 \frac{1}{\alpha\beta^2} \frac{5}{2m_1a_1^2}      
        \right )
    \end{equation}
    \begin{equation}
        U=
	    -\frac{\mu_0}{4\pi}
	    \frac{\gamma \mu_1^2}{2}
	    \frac{1}{r^3}(
	        \cos(\phi_1-\phi_2)
	        +3\cos(\phi_1+\phi_2 -2\theta)
	    )
    \end{equation}
\end{subequations}

If we go on to define our units as follows: 
$L_0=2a_1$ for length,
$F_0=3\mu_0 \mu_1^2/(2\pi L_0^4)$ for force,
$F_0L_0$ for energy,
$T_0=\sqrt{m_1L_0/F0}$ for time,
$\mu_1$ for magnetic moment,
$T_0^{-1},T_0^{-2}$ for angular velocities and accelerations,
$m_1L_0/T_0$ for linear momentum and 
$m_1L_0^2/T_0$ for angular momentum. This let's us rewrite the energies

\begin{subequations}
    \begin{equation}
        T=\frac{1}{2}\left (
	\frac{(1+\alpha)}{\alpha } p_r^2
	+\frac{(1+\alpha)}{\alpha } \frac{p_\theta^2}{r^2}
	+10 p_{\phi_1}^2 
	+\frac{10}{\alpha\beta^2} p_{\phi_2}^2      
        \right )
    \end{equation}
    \begin{equation}
        U=
	    -\frac{\gamma}{12}
	    \frac{1}{r^3}(
	        \cos(\phi_1-\phi_2)
	        +3\cos(\phi_1+\phi_2 -2\theta)
	    )
    \end{equation}
\end{subequations}

Let us finally make two more assumptions. First, that the two spheres are identical, $\alpha=\beta=\gamma=1$. Second, that there is some contact potential preventing the two sphere's from overlapping that prohibits r from getting less than 1. These two assumptions produce the hamiltonian we'll be investigating for the remainder of our paper

\begin{equation}
  \begin{multlined}
	H=T+U=
	\frac{1}{2}\left (
	2 p_r^2
	+2 \frac{p_\theta^2}{r^2}
	+10 p_{\phi_1}^2 
	+10 p_{\phi_2}^2      
        \right )
        \\
	-
	\frac{1}{12}
	\frac{1}{r^3}(
	        \cos(\phi_1-\phi_2)
	        +3\cos(\phi_1+\phi_2 -2\theta)
	    )+U_C
  \end{multlined}
\end{equation}

\section{Analysis}
\subsection{Analytic Results}

\subsubsection{Equilibrium Analysis}
Let us first examine the equations of motion and see if there exist any equilibria between the spheres while they are in contact with each other ($r=1$). The hamiltonian equations of motion we find are

\begin{subequations}
    \begin{equation}\label{fphi1}
        -\partial_{\phi_1} H = 
	- \frac{1}{12} \sin{\left (\phi_{1} - \phi_{2} \right )} - \frac{1}{4} \sin{\left (\phi_{1} + \phi_{2} - 2 \theta \right )}
    \end{equation}
    \begin{equation}\label{fphi2}
        -\partial_{\phi_2} H =
	\frac{1}{12} \sin{\left (\phi_{1} - \phi_{2} \right )} - \frac{1}{4} \sin{\left (\phi_{1} + \phi_{2} - 2 \theta \right )}
    \end{equation}
    \begin{equation}\label{ftht}
        -\partial_{\theta} H =
        \frac{1}{2} \sin{\left (\phi_{1} + \phi_{2} - 2 \theta \right )}
    \end{equation}
    \begin{equation}\label{fr}
        -\partial_{r} H = 
        2 p_{\theta}^{2} - \frac{1}{4} \cos{\left (\phi_{1} - \phi_{2} \right )} - \frac{3}{4} \cos{\left (\phi_{1} + \phi_{2} - 2 \theta \right )}
    \end{equation}
\end{subequations}

Inspecting equation \ref{ftht} we see immediately that orbital momentum is static if $\phi_{1} + \phi_{2} - 2 \theta = j \pi$ where j is any integer. Using that as a constraint we then see that spin momenta are both static if the previous equation holds and $\phi_{1} + \phi_{2} = k\pi$ where k is an integer independent of j. These two constraints produce a set of curves of equilibrium where

\begin{subequations}
	\begin{equation}
		\phi_1 = \frac{j+k}{2}\pi +\theta
	\end{equation}
	\begin{equation}
		\phi_2 = \frac{j-k}{2}\pi +\theta
	\end{equation}
\end{subequations}

all the forces on angular momenta are 0. However since our hamiltonian is periodic over $2\pi$ we're interested in (j,k) permuting through 0 and 1. This leaves us with 4 curves, until we consider the constraint that \ref{fr} must be negative to maintain contact. If j is odd, then we're left with a resulting radial force of $1/2$, and we lose contact. So we've narrowed down from an infinite number of equilibrium curves to two, which, using the j-k notation are 0-0 and 0-1.



\subsubsection{Normal Mode Analysis}
Equilibria and equations of motion in hand we can do small angle perturbations. Noting that near an equilibrium point 
$\Gamma_i = 
(
\phi_{1i},\phi_{2i},\theta_{i},
p_{\phi_{1i}},p_{\phi_{2i}},p_{\theta_{i}}
)
=
(\phi_{1i},\phi_{2i},\theta_{i},0,0,0)
$ 
the changes in momenta will be small we can do a multi-variable taylor expansion and produce the expression.

\begin{equation}\label{taylor_force}
	\dot p_n 
	=
	-\partial_{q_n}H
	\approx 
	\Sigma_m \partial_{q_m}(-\partial_{q_n} H)|_{\Gamma_i} q_m 
\end{equation}

If we define elements in a matrix 
$K_{n,m}= \partial_{q_m}(-\partial_{q_n} H)|_{\Gamma_i}$

We can rephrase \ref{taylor_force} as 

\begin{equation} \label{matrix_force}
	\vec{ \dot{ p }} \approx \widehat K \vec q
\end{equation}

To get this amenable to simple periodic solutions, we would like to recast the momentum vector in terms of a time derivative of our position.

\begin{equation}
	\dot{ q_n } = \partial_{p_n} H
\end{equation}

If we define elements in a diagonal matrix as 
$M_{n,n}=1/(\frac{d}{dt} \partial_{p_n} H)$
then we can rewrite \ref{matrix_force} as 

\begin{equation}
	\widehat M \vec{ \ddot{ q }} \approx \widehat K \vec q
\end{equation}

Assuming simple periodic solutions of the form $\vec q = \vec a e^{i\omega t}$ allows us to recast the above as

\begin{equation}
	( 
	\widehat K + \omega^2 \widehat M
	)  \vec{ a}e^{i\omega t} \approx 0
\end{equation}

If we treat the approximation as an equality, it will only hold for non-trivial motion if the determinant of the matrix $\widehat K - \omega^2 \widehat M$ (referred to here on as the perturbation matrix) is 0. The normal modes are the eigenvectors and values of the previous matrix.

For the 0,0 equilibrium we find the perturbation matrix to be
\begin{equation}
	\left[\begin{matrix}\frac{\omega^{2}}{10} - \frac{1}{3} & - \frac{1}{6} & \frac{1}{2}\\
	- \frac{1}{6} & \frac{\omega^{2}}{10} - \frac{1}{3} & \frac{1}{2}\\
	\frac{1}{2} & \frac{1}{2} & \frac{\omega^{2}}{2} - 1\end{matrix}\right]
\end{equation}

Which only has 2 non-zero eigen modes corresponding to when $\omega^2$ is equal to 5/3 and 7. The lower frequency mode has an eigenvector of [1, -1, 0], indicating it has no motion in the orbital angle, $\theta$, for this reason I refer to it as the spinning mode for it's sole form of motion. It possesses an interesting isomorphism we shall examine more later. The higher frequency's eigenvector is [5/2, 5/2,-1] indicating it does possess orbital motion and thus earns the moniker of orbital mode. Brief algebra will reveal that both modes have net-0 angular momentum. This likely stems from the result that net angular momentum for the system is constant.

For the 0,1 equilibrium the perturbation matrix is 
\begin{equation}
	\left[\begin{matrix}\frac{\omega^{2}}{10} - \frac{1}{6} & - \frac{1}{3} & \frac{1}{2}\\
	- \frac{1}{3} & \frac{\omega^{2}}{10} - \frac{1}{6} & \frac{1}{2}\\
	\frac{1}{2} & \frac{1}{2} & \frac{\omega^{2}}{2} - 1\end{matrix}\right]\end{equation}

Which merits two points of observation. First, all the same eigenvectors come forth along with 2 non-zero fundamental frequencies. Second, while the orbital mode has the same frequency, the spinning mode's $\omega^2=-5/3$ indicating that it is unstable.

Both the 0,0 and 0,1 equilibria points have a 


\subsubsection{Isomorphisms} Examining the spinning mode in more depth we present the variable substitution for the difference and sum of the dipole orientations and corresponding velocity

\begin{subequations}
	\begin{equation}
		\phi_d = \phi_1-\phi_2
	\end{equation}
	\begin{equation}
		\phi_t = \phi_1+\phi_2
	\end{equation}
	\begin{equation}
		\dot\phi_d = \dot\phi_1-\dot\phi_2
	\end{equation}
	\begin{equation}
		\dot\phi_t = \dot\phi_1+\dot\phi_2
	\end{equation}
\end{subequations}

Squaring and summing these new velocities we see find 
$\frac{1}{2}(\dot\phi_d^2 + \dot\phi_t^2) = \dot\phi_1^2+\dot\phi_2^2$
Which would mean that $p_{\phi_d}=\dot\phi_d/20$ and $p_{\phi_t}=\dot\phi_t/20$ which leads the contact hamiltonian in these coordinates to be

\begin{equation}
  \begin{multlined}
	H=T+U=
	\frac{1}{2}\left (
	2 p_\theta^2
	+20 p_{\phi_d}^2 
	+20 p_{\phi_t}^2      
        \right )
        \\
	-
	\frac{1}{12}
	(
	        \cos(\phi_d)
	        +3\cos(\phi_t-2\theta)
	    )
  \end{multlined}
\end{equation}

considering just the spinning mode where we begin with $\phi_t=\theta=p_{\phi_t}=p_\theta=0$ with this hamiltonian it becomes clear they all start at 0 and stay at 0, reducing this phase space from 6 dimensions to 2.

\begin{equation}
	H_{\phi_d}=
	10 p_{\phi_d}^2 
	-
	\frac{1}{12}
        \cos(\phi_d)
\end{equation}

Which is isomorphic with the hamiltonian to a simple pendulum. As expected, the hamilton equations lead us to  a 2nd order differential $\ddot \phi_d = - \frac{5}{3}\sin(\phi_d)$ providing a 2nd check that the frequency is correct.

\subsubsection{Coupling}

Considering now the Hamiltonian for the other two angles we have

\begin{equation}
	H_{\phi_t, \theta}=
	\frac{1}{2}\left (
	+2 p_\theta^2
	+20 p_{\phi_t}^2 
        \right )
	-
	\frac{1}{12}
	\frac{1}{r^3}(
	        3\cos(\phi_t-2\theta)
	    )
\end{equation}

Which holds while the contact criterion is observed.

\begin{subequations}
	\begin{equation}
		-\partial_{\phi_t}H= \dot p_{\phi_t} 
		= - \frac{1}{4} \sin{\left (\phi_{t} - 2 \theta \right )}
	\end{equation}
	\begin{equation}
		-\partial_{\theta}H= \dot p_{\theta}  
		= \frac{1}{2} \sin{\left (\phi_{t} - 2 \theta \right )}
		=-2\dot p_{\phi_t}
	\end{equation}
	\begin{equation}
		\partial_{p_{\phi_t}}H= \dot \phi_t 
		= 20 p_{\phi_t}
	\end{equation}
	\begin{equation}
		\partial_{p_{\theta}}H= \dot \theta 
		= 2 p_{\theta}
	\end{equation}
\end{subequations}

The rates of change for the momenta clearly denote a conserved quantity we'll denote as $L_0 = p_\theta + 2p_{\phi_t}$. With this we can produce the expression

\begin{equation}
  \begin{multlined}
	\phi_t(t) 
	= \phi_{t0} + \int_0^t  \dot \phi_t  dt
	= \phi_{t0} + \int_0^t  20 p_{ \phi_t } dt
	= \phi_{t0} + \int_0^t  10 (L_0-p_{ \theta }) dt
	\\
	= \phi_{t0} + 10L_0t -10 \int_0^t  p_{ \theta } dt
  \end{multlined}
\end{equation}

Which we'll rearrange to get

\begin{equation}\label{phi_int}
	\int_0^t  p_{ \theta } dt
	= L_0t + (\phi_{t0}  - \phi_t(t))/10
\end{equation}

similarly

\begin{equation}\label{tht_int}
  \begin{multlined}
	\theta(t) 
	= \theta_{ 0} + \int_0^t  \dot \theta  dt
	= \theta_{ 0} + \int_0^t  2 p_{\theta} dt
	\\
	\int_0^t  p_{\theta} dt = (\theta(t)-\theta_{0})/2
  \end{multlined}
\end{equation}

While the integral of $p_\theta$ is not generally analytic, the equality between the two holds which allows us to algebraically rearrange \ref{phi_int} and \ref{tht_int} to get

\begin{equation}
	\phi_t(t)=  -5(\theta(t)-\theta_{0}-\phi_{t0}/5) + 10L_0 t 
\end{equation}


\subsection{Numerical Results}
\subsubsection{Method}
\subsubsection{Periodic Solutions}
%needed diagrams;
%numerical values of period vs energy
%geometry layout/labels
%animatique of basic modes
%revtek4.1
\end{document}
