
\documentclass[12pt]{amsart}
\usepackage{geometry} % see geometry.pdf on how to lay out the page. There's lots.
\geometry{a4paper} % or letter or a5paper or ... etc
% \geometry{landscape} % rotated page geometry

% See the ``Article customise'' template for come common customisations

\title{Dynamics of Two Freely Rotating Dipoles}
\author{Peter Haugen}
%\date{} % delete this line to display the current date

%%% BEGIN DOCUMENT
\begin{document}

\maketitle
\tableofcontents

\section{Introduction}
A system comprised of two unrestricted spherical magnetic dipoles presents an interesting and fundamental set of phenomenon as two magnetic dipoles is amongst the simplest non-trivial macroscopic magnetic systems. Understanding their dynamics could prove useful in contexts such as the large dipole field proposed to provide a solar wind umbrella for mars.

\section{Describing the System}
Intrinsic properties of the two dipoles are: their respective radii, which we shall label as $a_1$ and $a_2$, their masses, $m_1$ and $m_2$, and magnitudes of their dipole moments, $\mu_1$ and $\mu_2$.
\subsection{Coordinates}
\subsubsection{Cartesian}

Each of the dipoles has a location $\vec r_1$ and $\vec r_1$ along with an orientation for the dipole $\phi_1$ and $\phi_2$, measured from the x-axis.
\subsubsection{Center Of Mass}
We can define a more appropriate set of coordinates and quantities using these fundamental descriptors. 

The total and reduced mass

\begin{equation}
m_t = m_1+m_2
\qquad
m_r = \frac{m_1m_2}{m_1+m_2}
\end{equation}

The center of mass 
\begin{equation}
\vec R = \frac{m_1 \vec r_1+m_2 \vec r_2}{m_t}
\end{equation}

And the displacement of dipole 2 from dipole 1
\begin{equation}
\vec r 
=  \vec r_2-\vec r_1 
= r (\cos(\theta) \hat x+\sin(\theta) \hat y)
\end{equation}


\subsection{Hamiltonian}
\subsubsection{Energies}
The kinetic energy, T, in the center of mass coordinates can be found to be
%citing taylor mechanics for the reduced kinetic
\begin{equation}
T = \frac{1}{2}(
	m_t \dot R^2
	+m_r \dot r^2
	+m_r r^2 \dot \theta^2
	+ I_1 \dot \phi_1^2
	+ I_2 \dot \phi_2^2
)
\end{equation}
The potential energy for dipole 1 is merely it's dot product with the local field, which in this case is caused entirely by dipole 2.
\begin{equation}
U_1 = -\vec \mu_1 \cdot \vec B_2
\end{equation}

With the magnetic field for a dipole at some point of interest $\vec r_i$ begin given by

\begin{equation}
\vec B_2 = 
\frac{\mu_0}{4\pi}(
	3\frac{\vec \mu_2 \cdot (\vec r_i - \vec r_2)}{|\vec r_i - \vec r_2|^5}(
	\vec r_i - \vec r_2
	)
	-\frac{\vec \mu_2}{|\vec r_i - \vec r_2|^3}
)
\end{equation}

Using our definition for $\vec r$ we can simplify the potential down to

\begin{equation}
U_1 = 
\frac{\mu_0}{4\pi}
\frac{1}{r^3}(
	\vec \mu_1 \cdot \vec \mu_2
	-3(
		\vec \mu_2 \cdot \hat r
		)(
		\vec \mu_1 \cdot \hat r		
		)
)
\end{equation}

This is obviously symmetric under substitution of 1 for 2 and vice versa so we have a total interaction potential of

\begin{equation}
U =
U_1+U_2
= 
2U_1
\end{equation}

Which when expanding the dot products out in terms of all three respective angles leaves us with

\begin{equation}
U(\phi_1, \phi_2, \theta, r) =
-\frac{\mu_0}{4\pi}
\frac{\mu_1 \mu_2}{2}
\frac{1}{r^3}(
	\cos(\phi_1-\phi_2)
	+3\cos(\phi_1+\phi_2 -2\theta)
)
\end{equation}

Having an expression for kinetic and potential energy, the lagrangian is as follows.

\begin{equation}
    L=T-U=
    \frac{1}{2}(
        m_t \dot R^2
        +m_r \dot r^2
        +m_r r^2 \dot \theta^2
        + I_1 \dot \phi_1^2
        + I_2 \dot \phi_2^2
    )
    +
    \frac{\mu_0}{4\pi}
    \frac{\mu_1 \mu_2}{2}
    \frac{1}{r^3}(
        \cos(\phi_1-\phi_2)
        +3\cos(\phi_1+\phi_2 -2\theta)
    )
\end{equation}

\subsubsection{Momenta}

The first observation we make is that the potential energy is independent of the center of mass coordinates making the corresponding momenta a constant of motion and we will proceed assuming we are in the inertial frame where it is 0 and R is 0. Determining the other momenta we use the relationship $\partial_{\dot q_i} L =p_i$ to arrive at

\begin{subequations}
    \begin{equation}
        \partial_{\dot \phi_1} L = I_1\dot \phi_1 = p_{\phi_1}
    \end{equation}
    \begin{equation}
        \partial_{\dot \phi_2} L =I_2\dot \phi_2 = p_{\phi_2}
    \end{equation}
    \begin{equation}
        \partial_{\dot \theta} L =m_r r^2 \dot \theta = p_{\theta}
    \end{equation}
    \begin{equation}
        \partial_{\dot r} L = m_r \dot r = p_r
    \end{equation}
\end{subequations}

Allowing us to find the Hamiltonian with the expression of 

\begin{equation}
H=l+\Sigma_i p_i \dot q_i
\end{equation}

\section{Analysis}
\subsection{Analytic Results}

\subsubsection{Equilibrium Analysis}
\subsubsection{Normal Mode Analysis}

\subsection{Numerical Results}
\subsubsection{Method}
\subsubsection{Periodic Solutions}

\end{document}
